\documentclass[12pt]{article}
\usepackage[english]{babel}
\usepackage{natbib}
\usepackage{url}
\usepackage[utf8x]{inputenc}
\usepackage{amsmath}
\usepackage{graphicx}
\graphicspath{{images/}}
\usepackage{parskip}
\usepackage{fancyhdr}
\usepackage{vmargin}
\setmarginsrb{3 cm}{2.5 cm}{3 cm}{2.5 cm}{1 cm}{1.5 cm}{1 cm}{1.5 cm}

\title{Disruptive Innovations in Healthcare}								% Title
\author{21111038}								% Author
\date{18 Feb 2022}											% Date

\makeatletter
\let\thetitle\@title
\let\theauthor\@author
\let\thedate\@date
\makeatother

\pagestyle{fancy}
\fancyhf{}
\rhead{\theauthor}
\lhead{\thetitle}
\cfoot{\thepage}

\begin{document}

%%%%%%%%%%%%%%%%%%%%%%%%%%%%%%%%%%%%%%%%%%%%%%%%%%%%%%%%%%%%%%%%%%%%%%%%%%%%%%%%%%%%%%%%%

\begin{titlepage}
	\centering
    \vspace*{0.5 cm}
    \includegraphics[scale = 0.10]{logo.jpg}\\[1.0 cm]	% University Logo
    \textsc{\LARGE  National Institute of Technology\newline\newline Raipur}\\[2.0 cm]	% University Name
	\textsc{\Large assignment 04}\\[0.5 cm]				% Course Code
	\rule{\linewidth}{0.2 mm} \\[0.4 cm]
	{ \huge \bfseries \thetitle}\\
	\rule{\linewidth}{0.2 mm} \\[1.5 cm]
	
	\begin{minipage}{0.4\textwidth}
		\begin{flushleft} \large
			\emph{Submitted To:}\\
			Saurabh Gupta\\
            Asst. Professor\\
            Department of Biomedical Engineering\\
			\end{flushleft}
			\end{minipage}~
			\begin{minipage}{0.4\textwidth}
            
			\begin{flushright} \large
			\emph{Submitted By :} \\
			Pradnya Manmode\\
            21111038\\
        First Semester\\
        Biomedical Engineering\\
		\end{flushright}
        
	\end{minipage}\\[2 cm]
	
	
    
    
    
    
	
\end{titlepage}


\title{ASSIGNMENT 04}
\maketitle

\indent
A new product, service, or business model is considered “disruptive” when it helps create a new market, eventually disrupting existing markets and displacing previous technologies.
\\
Disruptive innovation has happened over and over, in industries like computing, photography, telecommunications, and retail. Studies of these industries reveal two important facts about disruptive innovation:
1. The change resulting from disruptive innovation has always been positive, leaving us with better products and services than before.
2. It is usually new entrants into the market that figure out a better way of doing things.

\indent
Disruption is happening everywhere in healthcare – from AI to mHealth to 3D printing and robotics. Here are a few contemporary examples of disruptive healthcare technology:
\\
1. Consumer devices, wearables and apps – In the past, patients could only get biometric data when they went to the doctor’s office. Now health data gathered from smartwatches and mobile fitness trackers allow consumers to play a new role in their health journey.
\\
2. AI and machine learning – AI applications are everywhere in healthcare, from patient intake to predictive analytics to new drug develompment. This technology is changing how health systems operate and how care is delivered.\\
3. Telemedicine – COVID-19 accelerated the expansion of telemdicine, but with long-ranging impacts. Most patients say they are interested in virtual care going forward.\\
4. Blockchain – Blockchain is a database technology that stores data in a way that enhances security and usability. This innovation is changing many aspects of healthcare, including patient records, supply and distribution, and research.

\indent
How disruptive innovations happens:
\\
In addition to technological innovation, experts say there are two processes that often facilitate disruption in healthcare specifically.

1. Decentralization
The first is decentralization. Disruption typically involves new market entrants creating products that bring in new consumers. But in healthcare, everyone is already a consumer. In healthcare, disruption often shifts care from hospitals to clinics and office settings, and even into patients’ homes. Telemedicine is the most obvious example.

2. Transference of skills
The second process is matching clinician’s skill level with the difficulty of the medical problem. This means transferring skills from highly trained, expensive personnel, to more affordable providers, including technology-based care. This helps address expensive care due to overshoot of patient needs by health care institutions.

\indent

Some recent disruptive innovations in healthcare:
\\
1. IoT:
\\
The IoT is changing everything from patient records and monitoring to inventory control to preventative care. Communication and collaboration between healthcare professionals can now happen in real-time — an extremely beneficial aspect of the healthcare industry.
\\
2. EHRs:
\\
Electronic health records have gotten a facelift over the years. With the IoT, big data and devices’ connectivity provide up-to-date information about a patient at their point of care.
\\EHRs are easing communication between providers, as they are shareable through devices and secure networks. The convenience of EHRs is appreciated by healthcare professionals who can update and expand upon a patient’s medical history. EHRs can also further preventative care efforts and reduce medical errors by providing an easily readable, comprehensive look into a patient’s history.
\\
3. 3D Printing:
\\
3D printing impacts many industries in reducing labor costs while increasing production rates, and healthcare is one industry tapping into its enormous potential. Although initial prices may be high, 3D printing technology is developing rapidly every day, reducing the cost of manufacturing prototypes, prosthetics, tissue and skin, and even pharmaceuticals.3D printing is helping healthcare professionals and patients alike. Precise and custom designs can be made for patients who are not all the same regarding mass-produced prosthetics. 
\\
4. Augmented Reality (AI):
\\
Augmented reality is yet another emerging technology disrupting the healthcare industry—augmented reality supplements reality with images and sounds to create its own type of extended reality. AI can be used to show a patient exactly how to apply medication, wash and dress a wound, and other duties that can easily be done by a patient rather than a doctor to prevent further aggravation of the injury. Augmented reality will provide better education leading to preventative health and a reduction in medical error.
\\
5.Blockchain:
\\
Using blockchain, smart contracts can be created on a private ledger. Only a patient or healthcare professional qualified to see such data can cryptographically sign to access the data they need. For instance, patient records such as EHRs are celebrated for the convenience of sending back and forth, updated and expanding upon by professionals. Blockchain and smart contracts can be a critical step in securing this data, being easily accessible only when verified by the qualified people who need to access it.
\\
Although these technologies are relatively new, they explode on the scene and disrupt the industries they are implemented into. The healthcare industry recognizes these benefits and will capitalize on the technologies above.

\indent
Since these technologies are only recently being taken advantage of, healthcare has a long way to understand how they will integrate into operations fully. As healthcare grows, so will these technologies — disrupting healthcare in ways we may not even comprehend yet. One thing’s for sure: as healthcare becomes more intertwined with technology, we will only see the healthcare industry grow more optimized to provide quality health care that is easily accessible to all patients who require attention.

\end{document}
