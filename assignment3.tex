\documentclass[12pt]{article}
\usepackage[english]{babel}
\usepackage{natbib}
\usepackage{url}
\usepackage[utf8x]{inputenc}
\usepackage{amsmath}
\usepackage{graphicx}
\graphicspath{{images/}}
\usepackage{parskip}
\usepackage{fancyhdr}
\usepackage{vmargin}
\setmarginsrb{3 cm}{2.5 cm}{3 cm}{2.5 cm}{1 cm}{1.5 cm}{1 cm}{1.5 cm}

\title{Future of Healthcare}								% Title
\author{21111038}								% Author
\date{10 Feb 2022}											% Date

\makeatletter
\let\thetitle\@title
\let\theauthor\@author
\let\thedate\@date
\makeatother

\pagestyle{fancy}
\fancyhf{}
\rhead{\theauthor}
\lhead{\thetitle}
\cfoot{\thepage}

\begin{document}

%%%%%%%%%%%%%%%%%%%%%%%%%%%%%%%%%%%%%%%%%%%%%%%%%%%%%%%%%%%%%%%%%%%%%%%%%%%%%%%%%%%%%%%%%

\begin{titlepage}
	\centering
    \vspace*{0.5 cm}
    \includegraphics[scale = 0.20]{logo.jpg}\\[1.0 cm]	% University Logo
    \textsc{\LARGE  National Institute of Technology\newline\newline Raipur}\\[2.0 cm]	% University Name
	\textsc{\Large assignment 03}\\[0.5 cm]				% Course Code
	\rule{\linewidth}{0.2 mm} \\[0.4 cm]
	{ \huge \bfseries \thetitle}\\
	\rule{\linewidth}{0.2 mm} \\[1.5 cm]
	
	\begin{minipage}{0.4\textwidth}
		\begin{flushleft} \large
			\emph{Submitted To:}\\
			Saurabh Gupta\\
            Asst. Professor\\
            Department of Biomedical Engineering\\
			\end{flushleft}
			\end{minipage}~
			\begin{minipage}{0.4\textwidth}
            
			\begin{flushright} \large
			\emph{Submitted By :} \\
			Pradnya Manmode\\
            21111038\\
        First Semester\\
        Biomedical Engineering\\
		\end{flushright}
        
	\end{minipage}\\[2 cm]
	
	
    
    
    
    
	
\end{titlepage}


\title{ASSIGNMENT 03}
\maketitle

\indent

\title{\LARGE \textbf {Future of Healthcare} }
\maketitle

\indent
Like many other sectors, healthcare is about to enter a period of rapid change. Longevity and the advance of new technologies and discoveries – as well as innovative combinations of existing ones – are among the many factors propelling patient empowerment, which is fundamentally changing how we prevent, diagnose and cure diseases.

\indent
While advancements in medical knowledge and capability made over the years have been remarkable, hospitals have remained the same over the past fifty years.
In some ways, this is not so surprising. Hospitals are profoundly complex buildings, comprising of a wide range of services and units, from emergency rooms and operating theaters to clinical laboratories and imaging centers to food services and housekeeping. 

\indent
Tomorrow’s hospitals will have no doubt rely more heavily on robotics and digital technologies. Many of the physical and mental tasks that doctors perform today will be automated via hardware, software, and combinations of both. That will leave hospitals with more space in addition to the space being freed up through telemedicine and remote healthcare, which reduce the need for patient visits. The consequences will be far-reaching. Quality healthcare will become more accessible, as it will become cheaper, more efficient, and more convenient.
\\
Intermediate and longer-term care and rehabilitation centers would fill the remaining need. Artificially intelligent smart assistants-next generation  Siris and Alexas that can attend to basic everyday whims of patients will help with taking measurements and performing diagnostics. Robot carers will supervise and assist the elderly. 

\indent
While we may not be able to shrink human doctors down to microscopic size today, medicine on an atomic and molecular scale is fast becoming reality. Ever-smaller wearable devices that can monitor our vital signs are becoming common, but at the nanoscale, we could implant them into our bodies. Nanodevices could capture incredibly detailed data from deep within us, enabling doctors to personalize treatment. Such technology would radically improve medical imaging by delivering molecular resolution. Micromachines could identify and destroy cancer cells, keeping healthy cells untouched. With nanorobots, we could enter the body and even redesign the genome. 
\\
We can even imagine autonomous that eventually swim deep inside us detecting and reacting to problems as they arise.

\indent
One of the most hyped technological trends of the past ten years has been the promise of internet-connected everyday devices to transform everyday life by moving the internet off of our screens and into the physical environment. The ability to put cheap sensors into any objects, and connect them to the internet, offers a host of benefits, particularly in healthcare. Patients save time by being monitored remotely.  Doctors also receive continuous patient data, giving them a more detailed picture of their patients’ health.
Huge, anonymized datasets are already being offered to researchers to help improve the efficacy of drugs and better understand the development of cancer, among other afflictions. 

\indent
It is only very recently, with the advent of deep-learning algorithms in the current AI renaissance, that machines have started to equal or at times exceed humans in a wide range of perceptual tasks. This has truly transformative potential in medicine.
Radiology and pathology are two examples of medical specializations that are at their core about spotting patterns and taking readings. This is significant as these specialties alone make up a quarter of a typical health budget. Soon, the world’s best diagnostician in most medical specialties from not just radiology and pathology, but on to oncology, dermatology, ophthalmology- will be an algorithm.

\indent
Innovation trends in healthcare point towards a future where our health is monitored and provided continuously, wherever we are, with less and less need for bulky physical infrastructure.
\\
In medicine and healthcare, digital technology could help transform unsustainable healthcare systems into sustainable ones, equalize the relationship between medical professionals and patients, provide cheaper, faster and more effective solutions for diseases.

\end{document}