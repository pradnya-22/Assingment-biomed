\documentclass[12pt]{article}
\usepackage[english]{babel}
\usepackage{natbib}
\usepackage{url}
\usepackage[utf8x]{inputenc}
\usepackage{amsmath}
\usepackage{graphicx}
\graphicspath{{images/}}
\usepackage{parskip}
\usepackage{fancyhdr}
\usepackage{vmargin}
\setmarginsrb{3 cm}{2.5 cm}{3 cm}{2.5 cm}{1 cm}{1.5 cm}{1 cm}{1.5 cm}

\title{Solutions to Covid-19 provided by Biomedical Engineers}								% Title
\author{21111038}								% Author
\date{4 March 2022}											% Date

\makeatletter
\let\thetitle\@title
\let\theauthor\@author
\let\thedate\@date
\makeatother

\pagestyle{fancy}
\fancyhf{}
\rhead{\theauthor}
\lhead{\thetitle}
\cfoot{\thepage}

\begin{document}

%%%%%%%%%%%%%%%%%%%%%%%%%%%%%%%%%%%%%%%%%%%%%%%%%%%%%%%%%%%%%%%%%%%%%%%%%%%%%%%%%%%%%%%%%

\begin{titlepage}
	\centering
    \vspace*{0.5 cm}
    \includegraphics[scale = 0.17]{logo.jpg}\\[1.0 cm]	% University Logo
    \textsc{\LARGE  National Institute of Technology\newline\newline Raipur}\\[2.0 cm]	% University Name
	\textsc{\Large assignment 06}\\[0.5 cm]				% Course Code
	\rule{\linewidth}{0.2 mm} \\[0.4 cm]
	{ \huge \bfseries \thetitle}\\
	\rule{\linewidth}{0.2 mm} \\[1.5 cm]
	
	\begin{minipage}{0.4\textwidth}
		\begin{flushleft} \large
			\emph{Submitted To:}\\
			Saurabh Gupta\\
            Asst. Professor\\
            Department of Biomedical Engineering\\
			\end{flushleft}
			\end{minipage}~
			\begin{minipage}{0.4\textwidth}
            
			\begin{flushright} \large
			\emph{Submitted By :} \\
			Pradnya Manmode\\
            21111038\\
        First Semester\\
        Biomedical Engineering\\
		\end{flushright}
        
	\end{minipage}\\[2 cm]
	
	
    
    
    
    
	
\end{titlepage}


\title{ASSIGNMENT 06}
\maketitle

\indent
The coronavirus infectious disease (COVID-19) pandemic emerged at the end of 2019, and was caused by the Severe Acute Respiratory Syndrome Coronavirus 2 (SARS-CoV-2), which has resulted in an unprecedented health and economic crisis worldwide. One key aspect, compared to other recent pandemics, is the level of urgency, which has started a race for finding adequate answers. Solutions for efficient prevention approaches, rapid, reliable, and high throughput diagnostics, monitoring, and safe therapies are needed. Research across the world has been directed to fight against COVID-19. Biomedical science has been presented as a possible area for combating the SARS-CoV-2 virus due to the unique challenges raised by the pandemic, as reported by epidemiologists, immunologists, and medical doctors, including COVID-19’s survival, symptoms, protein surface composition, and infection mechanisms.
\\
The use of medical devices in the COVID pandemic is the unfortunate indication that the patients are displaying severe respiratory distress symptoms and need a form of assistance to breathe.

\\
{\large Oxygen:}
\\
The first form for mild respiratory insufficiency is usually the supply of extra oxygen through a nasal cannula or a more intrusive face mask. Most of the time, the oxygen comes in the form of cylinders, either small for portability or large for stationary patients and longer-term supply.\\
Oxygen concentrators represent an attractive alternative to tanks although this is not typically in use while caring for COVID-19 patients in hospital settings. Oxygen concentrators extract oxygen from the air on demand and supply it directly to the patient. Concentrators come in a variety of sizes from a portable shoulder bag form factor, to higher capacity stationary machines for patients who need oxygen 24/7.

Variants of oxygen supply include high flow nasal oxygen (HFNO) which delivers warmed and humidified oxygen, to avoid the drying of airways, at high flow rates - typically tens of litres/min) at body temperature and up to 100% RH and 100% oxygen.

\newpage

{ \large Continuous Positive Airway Pressure (CPAP):}

\\
The next step up in treating COVID-19 patients is Continuous Positive Airway Pressure (CPAP) which is initially intended to prevent airways collapse in sleep apnoea patients, but has been shown to be beneficial to COVID patients if applied early enough in the progression of the disease.

A well-fitted face mask is an essential component of a CPAP system and as such it is quite intrusive. CPAP is only appropriate for patients who are capable of some breathing strength as CPAP effectively opposes some resistance to expiration. Variants exist that either automatically adjust the level of pressure to the patients breathing characteristics (APAP) or have different levels of pressure for inspiration and expiration (BiPAP). CPAP usually supplies (filtered) air to the patient but most masks have a port for supplementing the supply with oxygen.

\indent

{ \large Ventilators:}
\\
Patients who cannot breathe spontaneously need to be put on a ventilator. Ventilators are capable of replacing the breath function and patients in an advanced state of respiratory distress are usually intubated and sedated at the beginning of the treatment.
\\
Ventilators are capable of replacing the breath function and patients in an advanced state of respiratory distress are usually intubated and sedated at the beginning of the treatment. They are complex systems providing the healthcare professionals with a lot of flexibility to adapt the assisted breathing settings and to be able to wean recovering patients off the ventilator gradually.
\\
Modern ventilators are typically closed loop pressure controlled and capable of detecting spontaneous breathing to synchronise assistance for recovering patients. They also enable the control of the composition of the gas the patient breathes from normal air to 100 percent oxygen, usually taking their supply from the hospital’s gas supply network but can also be coupled to oxygen tanks or oxygen concentrators if used in a setting where there is no gas network.

\indent

{\large Protective Personal Masks}
\\
The reduction of the coronavirus persistence on PPE surfaces can be accomplished by preventing the adhesion of the respiratory droplets onto the material surfaces. To this end, metal oxides have been employed on different types of surfaces to induce biocidal effects over pathogens. Copper oxide (CuO) is one of the compounds that has been used to design PPE with biocidal properties, such as respiratory face masks. This type of equipment is the first line of defense to reduce the spread of respiratory viruses according to the WHO. 
This PPE showed enhanced superhydrophobic performances providing better protection toward incoming respiration droplets which can be the virus vector. These devices have been fabricated by functionalizing commercially available masks with graphene nanosheets and exhibited unprecedented self-cleaning and photothermal properties owing to the intrinsic physicochemical properties provided by graphene. This might result in high economic and environmental costs/benefits impact worldwide. All of these findings evidence that protective equipment, such as face masks worn by healthcare workers and the lay public, can benefit from nanomaterials, by incorporating an additional line of defense against the surface- and aerosol-persistent pathogens.

\indent

{\large Surface Decontamination:}
\\
In addition to recommended PPE for caregivers and patients, hand hygiene and surface decontamination are also key to health safety, as the SARS-CoV-2 is known to remain viable on surfaces for hours to days. Disinfection practices should be focused on meticulous hygiene in workspaces to minimize contaminated surfaces. As mentioned previously, the nosocomial spread has been documented for infectious pathogens, including coronaviruses. The multifunctionality of nanomaterials endow them with the feasibility of being coated on a large number of common In addition to being cost-effective and easy to synthesize, copper oxide (CuO) NPs exhibit interesting biological properties [134]. For these reasons, they have been incorporated successfully for microorganism and virus inactivation purposes on contaminated surfaces. Thus, CuO NPs will play an important role to promote the risk reduction of people exposed to COVID-19 virus spread.surfaces to act as biocidal agents for different types of pathogens, including viruses.
\\
1.Silver Nanoparticles:
\\
In this regard, silver nanoparticles (AgNPs) have been used as an antiseptic and disinfectant as they are able to interact with the disulfide bonds of the glycoprotein/protein contents of human pathogens, such as viruses, bacteria, and fungi, and induce their cell lysis.
\\
2.Copper Oxide Nanoparticles:
\\
In addition to being cost-effective and easy to synthesize, copper oxide (CuO) NPs exhibit interesting biological properties. For these reasons, they have been incorporated successfully for microorganism and virus inactivation purposes on contaminated surfaces. Thus, CuO NPs will play an important role to promote the risk reduction of people exposed to COVID-19 virus spread.
\\
3. Titanium Oxide Nanoparticles:
\\
Titanium dioxide (TiO2) has also been studied owing to its biocidal properties to reduce contamination (e.g., biofouling) on various types of surfaces and media. Although many studies have focused on its disinfectant effects on bacteria, other studies have addressed the virus inactivation by TiO2, particularly at the nanoscale. Specifically, the high photocatalytic property of TiO2 NPs has been explored to inactivate enveloped and non-enveloped viruses on different surfaces and media.
\\
4. Graphene:
\\
Associated with different materials, graphene sheets have been investigated for coating applications, such as in medical devices , due to their mechanical and biocidal properties. Specifically, graphene oxide (GO) and reduced graphene oxide (rGO) sheets have been used to coat surfaces and films to deactivate pathogens including human viruses. The antiviral effect of GO has been confirmed on other viruses, such as herpes simplex virus type 1, feline coronavirus, and infectious bursal disease virus, which have been deactivated by sulfonated magnetic NPs functionalized with rGOand GO-AgNPs respectively. These results demonstrate that graphene can be used in nanocomposites or alone to impart antiviral characteristics for surface coating applications.

\indent

{\large Artificial Intelligence:}
\\
Artificial intelligence (AI) allows to analyze the immense amount of information generated every second to predict the behavior of the infection in the world by monitoring millions of variables related to the coronavirus from mathematical models, projections, new outbreaks, the behavior of industry-specific shoppers, web searches for people with COVID-19 symptoms, worldwide clinical trials and real-time news. The Canadian AI company, BlueDot, has several software products capable of predicting abnormal outbreaks of diseases with infectious potential of massive infections throughout the world. This company was able to alert their clients of an abnormal increase in pneumonia cases in Wuhan 9 days before the WHO warned the world about COVID-19. Big data management allows the real time monitoring of the outbreak. The information available today in combination with real-time information in the location of people and mathematical models are the perfect combination to generate predictions that help protecting the most vulnerable and making data-driven political decisions.

\indent

Like these few, there are many more inventions that help us to fight against harmful diseases covid-19.

\\
The last decades have witnessed increasing interest and tremendous developments in biomedical science and engineering through adaptative, collaborative, multidisciplinary, and innovative research efforts made by private and public institutions to meet the challenges that constantly arise including the current, unprecedented COVID-19 pandemic; this has significantly benefited human health and society. For example, efficient and effective monitoring, detection and prevention of viral outbreaks, along with the development of treatments and vaccines, have been achieved; this great progress has enabled to save countless lives.




\end{document}
